\documentclass[11pt]{article}

\usepackage{amsmath}
\usepackage{amssymb}
\usepackage{fancyhdr}
\usepackage[margin=1in]{geometry}
\usepackage{microtype}
\usepackage{graphicx}
\usepackage{tikz}
\usepackage{mathtools}
\graphicspath{ {images/} }

\DeclarePairedDelimiter{\ceil}{\lceil}{\rceil}
\newcommand{\topic}[1] {\vspace{.25in} \hrule\vspace{0.5em}
  \noindent{\bf #1} \vspace{0.5em}
  \hrule \vspace{.10in}}
\renewcommand{\part}[1] {\vspace{.10in} {\bf (#1)}}

\newcommand{\myname}{Kyle Vedder}
\newcommand{\myaddress}{kvedder@umass.edu}

\setlength{\parindent}{0pt}
\setlength{\parskip}{5pt plus 1pt}

\pagestyle{fancyplain}
\lhead{\fancyplain{}{\textbf{CS-603 Particle Filter}}}
\rhead{\fancyplain{Kyle Vedder}{\myname\\ \myaddress}}
\chead{\fancyplain{}{}}
\usepackage{color}

\newenvironment{answer}{\color{blue}\ttfamily}{\par}

\begin{document}

\medskip 
\thispagestyle{plain}
\begin{center}
  {\Large CS-603 Particle Filter} \\
  \myaddress\\
  \today \\
\end{center}

\topic{Background}

The goal of this assignment was to implement a particle filter to do
localization of a robot, given a map as well as real odometry and laser scan
data. The challenges faced are that neither the laser scan nor the odometry are
noise free, and thus treating them as the ground truth will lead to incorrect
localization.

For this specific implementation, odometry is run at 20hz and laser scans are
run at 10hz. In addition, odometry data is particularly noisey,
meaning that the distribution of possible locations after an odometry move is
fairly large, and that without the aid of laser data to facilitate convergence
of possible locations, the uncertainty of positions can grow to be rather large.

\topic{Motion Model}

The motion model captures the uncertainty in the movement of the robot. For
example, if the robot reports moving forward $0.1m$

\end{document}
